% ghcmodHappyHaskellProgram-Kg.tex
\begin{hcarentry}[updated]{ghc-mod --- Happy Haskell Programming}
\report{Kazu Yamamoto}%11/13
\status{open source, actively developed}
\makeheader

{\tt ghc-mod} is a backend command to enrich Haskell programming on editors including Emacs, Vim and Sublime. The ghc-mod package on Hackage includes the {\tt ghc-mod} command and Emacs front-end. 

Emacs front-end provides the following features:

\begin{description}
\item[Completion] You can complete a name of keyword, module, class, function, types, language extensions, etc.

\item[Code template] You can insert a code template according to the position of the cursor. For instance, ``module Foo where'' is inserted in the beginning of a buffer.

\item[Syntax check] Code lines with error messages are automatically highlighted thanks to flymake. You can display the error message of the current line in another window. {\tt hlint} %\cref{hlint} 
can be used instead of GHC to check Haskell syntax. 

\item[Document browsing] You can browse the module document of the current line either locally or on Hackage.

\item[Expression type] You can display the type/information of the expression on the cursor. 
\end{description}

\noindent
There are two Vim plugins:
\begin{itemize}
\item ghcmod-vim
\item syntastic
\end{itemize}

\noindent
Here are new features:
\begin{itemize}
\item {\tt ghc-mod} now supports the sandbox of {\tt cabal} 1.18 and stopped supporting the sandbox of {\tt cabal-dev}.
\item The ghc-mod package also provides a library now.
\end{itemize}

\FurtherReading
  \url{http://www.mew.org/~kazu/proj/ghc-mod/en/}
\end{hcarentry}
